
\documentstyle[12pt]{article}
%%%%%%%%%%%%%%%%%%%%%%%%%%%%%%%%%%%%%%%%%%%%%%%%%%%%%%%%%%%%%%%%%%%%%%%%%%%%%%%%%%%%%%%%%%%%%%%%%%%%%%%%%%%%%%%%%%%%%%%%%%%%
%TCIDATA{Created=Wed Jul 31 13:00:10 1996}
%TCIDATA{LastRevised=Tue Nov 12 22:45:18 1996}
%TCIDATA{Language=American English}

\textwidth 17true cm
\textheight 8.65true in
\oddsidemargin -0.05true in
\evensidemargin 0.25true in
\topmargin -0.5true in
\headsep 0.4true in
\renewcommand{\baselinestretch}{1.5}
\input{tcilatex}
\begin{document}

\title{ANALYTIC FIRST DERIVATIVES FOR EXPLICITLY CORRELATED, MULTI-CENTER, GAUSSIAN
GEMINALS}
\author{D. Gilmore, Pawel M. Kozlowski\thanks{%
Department of Chemistry, Princeton University, Princeton, New Jersey 08544},
D. B. Kinghorn\ and L. Adamowicz \\
%EndAName
Department of Chemistry, University of Arizona, Tucson AZ 85721.}
\date{(Received...........................................................)}
\maketitle

\begin{abstract}
Variational calculations utilizing the analytic gradient of explicitly
correlated Gaussian molecular integrals are presented for the ground state
of the hydrogen molecule. Preliminary results serve to motivate the need for
general formulae for analytic first derivatives of molecular integrals
involving multi-center, explicitly correlated Gaussian geminals with respect
to Gaussian exponents and coordinates of the orbital centers. Explicit
formulae for analytic first derivatives of Gaussian functions containing
correlation factors of the form $\exp (-\beta r_{ij}^2)$ are derived and
discussed.
\end{abstract}

\newpage

\section{Introduction}

The idea of including explicitly interparticular distances in variational
wavefunctions has a long history in quantum chemistry. It appears in the
early work of Hylleraas \cite{hyl29}, James and Coolidge \cite{jam33} and in
some of the most accurate studies for the $H_2$ molecule by Kolos and
Wolniewicz \cite{kol65}. It is generally accepted that Hylleraas-type
expansions give the most accurate description of two- and three-electron
systems, as well as reproducing the electron-cusp correctly. The difficulty
of extending the Hylleraas expansion for larger systems lies in the
complexity of the required integrals, both in their analytical derivation
and computational implementation (see, for example \cite{fromm1, king1}).
The basis set of explicitly correlated Gaussians (ECG) proposed by Boys \cite
{boy60} and Singer \cite{sin60} mitigate this problem with all the required
integrals having closed analytic forms \cite{les64, koz91, koz92a}. Since
the introductory work of Boys and Singer, basis sets of ECG functions, in
particular Gaussian geminals (GG), have been applied to many different
chemically and physically significant problems. An extensive bibliography on
ECG's may be found in references \cite{sza83} and \cite{koz91}. Very recent
applications constitute the nonadiabatic studies of small systems \cite
{koz93a, koz93} and accurate multiple-moment calculations for $H_2$ and $D_2$
\cite{kom93}. In all applications of GG's, careful optimization of the
non-linear parameters is essential to obtain high-quality results. Although
ECG's do not possess correct behavior near singularities, one may expect
that long and well-optimized expansions should reproduce the cusp reasonably
well. Recent studies presented by Cencek and Rychlewski \cite{cen93} support
this contention and place application of ECG's in a new perspective. These
authors use a wavefunction expanded in terms of ECG's for the ground state
of $H_2$ at $R=1.4011a.u.$ internuclear separation. When this molecular
system is adjusted to $R=1.4a.u.$, one obtains an energy better than the one
corresponding to the generalized James and Coolidge expansion. Liu and
Hagstrom \cite{liu93}, and more recently Kolos \cite{kol94}, have suggested
that the generalized James-Coolidge expansion may not have been sufficiently
flexible and therefore would not have converged to the true Born-Oppenheimer
energy.

These results encourage further study of ECG's and show that the quality of
results obtained can be comparable to Hylleraas-type implementations.
Moreover, with the relatively simple evaluation of molecular, two-electron
integrals, application of ECG's does not suffer from restriction to
few-electron systems as methods employing Hylleraas functions do.

In most applications of GG's, or other types of explicitly correlated
Gaussians, the optimization of the non-linear parameters has been achieved
using numerical methods such as Powell's conjugate gradient method \cite
{powell1}. Other optimization methods have been based on tempering
procedures \cite{poshusta1, Alexander1, Alexander2, Rybak} where the
Gaussian exponents were generated as a sequence of parameters with only a
few of the variables being optimized. It is surprising that standard
optimization techniques based on analytical derivatives, widely used in
other areas of quantum chemistry, were only just applied to correlated
Gaussians quite recently \cite{koz92a, koz92b, kinghorn1}. In our previous
works, we developed analytical first- and second-derivatives in conjunction
with non-adiabatic, few-body calculations \cite{koz92a, koz92b}. In our
present work, we continue this research direction by considering the problem
of calculating first derivatives of multi-center, two-electron integrals for
use in variational Born-Oppenheimer calculations.

The object of this paper is to present the derivation of analytical first
derivatives for multi-center GG's and to propose the possibility of using
them in variational calculations. The existing code for evaluation of
integrals can be easily modified to obtain appropriate formulae for
analytical derivatives of the integrals with respect to the parameters in
the basis functions. The final section of this paper contains preliminary
numerical calculations for the ground state of the hydrogen molecule
utilizing analytic first derivatives of the Rayleigh energy functional with
respect to linear and non-linear parameters of the wavefunction.

\section{Basic properties of Gaussian Geminals}

Analytical differentiation of an energy functional requires the evaluation
of appropriate derivatives of molecular integrals with respect to the
non-linear parameters. Three different type of derivatives will be
considered: those with respect to orbital exponents, correlation exponents
and orbital centers. In this section we discuss basic properties of GG that
are important for the derivation of analytical first derivatives. It should
be pointed out that analytical derivatives, expressed in terms of GG, can be
rather long and complicated. We would like to keep the main body of the
paper as simple as possible; to that extent, let us introduce the following
notation for a one-dimensional Cartesian Gaussian: 
\begin{equation}
g(l^\mu ,x,R_x^\mu ,\alpha ^\mu )=(x-R_x^\mu )^{l^\mu }\exp (-\alpha ^\mu
|x-R_x^\mu |^2),
\end{equation}
thus, for a general Cartesian Gaussian we have, 
\begin{equation}
G(l^\mu ,m^\mu ,n^\mu ,{\bf r},{\bf R}^\mu ,\alpha ^\mu )=g(l^\mu ,x,R_x^\mu
,\alpha ^\mu )g(m^\mu ,y,R_y^\mu ,\alpha ^\mu )g(n^\mu ,z,R_z^\mu ,\alpha
^\mu ),
\end{equation}
and finally for a Cartesian GG, 
\[
\phi (l_1^\mu ,m_1^\mu ,n_1^\mu ,l_2^\mu ,m_2^\mu ,n_2^\mu ;{\bf R}_1^\mu ,%
{\bf R}_2^\mu )=
\]
\begin{equation}
G(l_1^\mu ,m_1^\mu ,n_1^\mu ,{\bf r}_1,{\bf R}_1^\mu ,\alpha _1^\mu
)G(l_2^\mu ,m_2^\mu ,n_2^\mu ,{\bf r}_2,{\bf R}_2^\mu ,\alpha _2^\mu )\exp
(-\beta _{12}^\mu {\bf r}_{12}^2).  \label{eq3}
\end{equation}
Alternatively, the last equation can be rewritten as a product of three
two-dimensional GG's: 
\[
\phi (l_1^\mu ,m_1^\mu ,n_1^\mu ,l_2^\mu ,m_2^\mu ,n_2^\mu ;{\bf R}_1^\mu ,%
{\bf R}_2^\mu )=
\]
\begin{equation}
\phi (l_1^\mu ,l_2^\mu ;R_{1,x}^\mu ,R_{2,x}^\mu )\phi (m_1^\mu ,m_2^\mu
;R_{1,y}^\mu ,R_{2,y}^\mu )\phi (n_1^\mu ,n_2^\mu ;R_{1,z}^\mu ,R_{2,z}^\mu )%
\text{.}
\end{equation}
where, for example, the {\it x}-component is given as,
\[
\phi (l_1^\mu ,l_2^\mu ;R_{1,x}^\mu ,R_{2,x}^\mu )=g(l_1^\mu
,x_1,R_{1,x}^\mu ,\alpha _1^\mu )g(l_2^\mu ,x_2,R_{2,x}^\mu ,\alpha _2^\mu
)\exp \left[ -\beta _{12}^\mu {\bf (}x_1-x_2)^2\right] .
\]

The general expression for analytical derivatives of molecular integrals
with respect to orbital exponents, correlation exponents or orbital centers
can be obtained by direct differentiation of general formulae for GG
integrals. Before we demonstrate this expressly, let us examine the result
of differentiating a GG with respect to non-linear parameters. Examination
of the result of differentiation of the GG with respect to orbital
exponents, which can be expressed in terms of products involving
one-dimensional Gaussians, yields: 
\[
{\frac{{\partial }}{{\ \partial \alpha _1^\mu }}}\phi (l_1^\mu ,m_1^\mu
,n_1^\mu ,l_2^\mu ,m_2^\mu ,n_2^\mu ;{\bf R}_1^\mu ,{\bf R}_2^\mu )\ =
\]
\[
-\ [g(l_1^\mu +2,x_1,R_{1x}^\mu ,\alpha _1^\mu )g(m_1^\mu ,y_1,R_{1y}^\mu
,\alpha _1^\mu )g(n_1^\mu ,z_1,R_{1z}^\mu ,\alpha _1^\mu )\ +
\]
\[
g(l_1^\mu ,x_1,R_{1x}^\mu ,\alpha _1^\mu )g(m_1^\mu +2,y_1,R_{1y}^\mu
,\alpha _1^\mu )g(n_1^\mu ,z_1,R_{1z}^\mu ,\alpha _1^\mu )\ +
\]
\[
g(l_1^\mu ,x_1,R_{1x}^\mu ,\alpha _1^\mu )g(m_1^\mu ,y_1,R_{1y}^\mu ,\alpha
_1^\mu )g(n_1^\mu +2,z_1,R_{1z}^\mu ,\alpha _1^\mu )]\times 
\]
\begin{equation}
G(l_2^\mu ,m_2^\mu ,n_2^\mu ,{\bf r}_2,{\bf R}_2^\mu ,\alpha _2^\mu )\exp
(-\beta _{12}^\mu {\bf r}_{12}^2).
\end{equation}
The derivative with respect to correlation exponents 
\begin{equation}
{\frac{{\partial }}{{\ \partial \beta _{12}^\mu }}}\phi (l_1^\mu ,m_1^\mu
,n_1^\mu ,l_2^\mu ,m_2^\mu ,n_2^\mu ;{\bf R}_1^\mu ,{\bf R}_2^\mu )\
=-r_{12}^2\phi (l_1^\mu ,m_1^\mu ,n_1^\mu ,l_2^\mu ,m_2^\mu ,n_2^\mu ;{\bf R}%
_1^\mu ,{\bf R}_2^\mu ),
\end{equation}
has a rather simple form. This becomes a little more complicated when one
wishes to replace the right hand side of the last equation by the
appropriate combination of Cartesian Gaussians. To demonstrate this, recall
that $xg(l^\mu ,x,R_x^\mu ,\alpha ^\mu )$ can be rewritten as a sum of two
Gaussians. 
\begin{equation}
xg(l^\mu ,x,R_x^\mu ,\alpha ^\mu )=g(l^\mu +1,x,R_x^\mu ,\alpha ^\mu
)+R_x^\mu g(l^\mu ,x,R_x^\mu ,\alpha ^\mu ).
\end{equation}
Multiple application of the above operation to the derivative with respect
to correlation exponents leads to the following (for brevity, we demonstrate
only the $x$-component of this expression): 
\[
(x_2-x_1)^2\phi (l_1^\mu ,m_1^\mu ,n_1^\mu ,l_2^\mu ,m_2^\mu ,n_2^\mu ;{\bf R%
}_1^\mu ,{\bf R}_2^\mu )\ =
\]
\[
\bigg[
\phi (l_1^\mu +2,l_2^\mu ;R_{1,x}^\mu ,R_{2,x}^\mu )\ -\ 2\phi (l_1^\mu
+1,l_2^\mu +1;R_{1,x}^\mu ,R_{2,x}^\mu )\ +\ \phi (l_1^\mu ,l_2^\mu
+2;R_{1,x}^\mu ,R_{2,x}^\mu )
\]
\[
\ +\ 2(R_{1,x}^\mu -R_{2,x}^\mu )\phi (l_1^\mu +1,l_2^\mu ;R_{1,x}^\mu
,R_{2,x}^\mu )\ +\ 2(R_{2,x}^\mu -R_{1,x}^\mu )\phi (l_1^\mu ,l_2^\mu
+1;R_{1,x}^\mu ,R_{2,x}^\mu )\ +\ 
\]
\begin{equation}
\ +\ (R_{1,x}^\mu -R_{2,x}^\mu )^2\phi (l_1^\mu ,l_2^\mu ;R_{1,x}^\mu
,R_{2,x}^\mu )\bigg] \phi (m_1^\mu ,m_2^\mu ;R_{1,y}^\mu ,R_{2,y}^\mu )\phi
(n_1^\mu ,n_2^\mu ;R_{1,z}^\mu ,R_{2,z}^\mu )
\end{equation}
The last derivative, to be considered in this section, is that with respect
to the orbital centers: 
\[
{\frac{{\partial }}{{\ \partial R_{1,x}^\mu }}}\phi (l_1^\mu ,m_1^\mu
,n_1^\mu ,l_2^\mu ,m_2^\mu ,n_2^\mu ;{\bf R}_1^\mu ,{\bf R}_2^\mu )\ =
\]
\[
\lbrack 2\alpha _1^\mu g(l_1^\mu +1,x_1,R_{1y}^\mu ,\alpha _1^\mu )-l_1^\mu
g(l_1^\mu -1,x_1,R_{1x}^\mu ,\alpha _1^\mu )]\times 
\]
\begin{equation}
g(m_1^\mu ,y_1,R_{1y}^\mu ,\alpha _1^\mu )g(n_1^\mu ,z_1,R_{1z}^\mu ,\alpha
_1^\mu )G(l_2^\mu ,m_2^\mu ,n_2^\mu ,{\bf r}_2,{\bf R}_2^\mu ,\alpha _2^\mu
)\exp (-\beta _{12}^\mu {\bf r}_{12}^2)
\end{equation}

Let us summarize some of properties of interest for GG's which are helpful
for the generation of analytical derivatives. Let ${\bf R}^{\mu \nu }$
denote the third common center of two Gaussians as: 
\begin{equation}
{\bf R}^{\mu \nu }={\frac{\alpha ^\mu {\bf R}^\mu +\alpha ^\nu {\bf R}^\nu }{%
\alpha ^\mu +\alpha ^\nu }}.
\end{equation}
The product of two Gaussian can be expressed as: 
\begin{equation}
\phi (0,0,0,0;{\bf R}_1^\mu ,{\bf R}_2^\mu )\phi (0,0,0,0;{\bf R}_1^\nu ,%
{\bf R}_2^\nu )=K_{\mu \nu }\phi (0,0,0,0;{\bf R}_1^{\mu \nu },{\bf R}%
_2^{\mu \nu }),
\end{equation}
with $K_{\mu \nu }$ represented as, 
\begin{equation}
K_{\mu \nu }=\exp \left( -\sum_{i=1}^2{\frac{\alpha _i^\mu \alpha _i^\nu }{%
\alpha _i^\mu +\alpha _i^\nu }}|{\bf R}_i^\mu -{\bf R}_i^\nu |^2\right) .
\label{k}
\end{equation}
For a general Cartesian Gaussian, expansion of the polynomial of coordinates
and center positions with respect to the common centers yields: 
\begin{equation}
(x_p-R_p^\mu )^{l_p^\mu }(x_p-R_p^\nu )^{l_p^\nu }=\sum_i^{l_p^\mu +l_p^\nu
}f_i(l_p^\mu ,l_p^\nu ,\overline{R_{p,x}^{\mu \nu }R_{p,x}^\mu },\overline{%
R_{p,x}^{\mu \nu }R_{p,x}^\nu })(x_p-R_{p,x}^{\mu \nu })^i,
\end{equation}
with 
\begin{equation}
f_j(l,m,a,b)=\sum_{i=\max \left\{ 0,j-m\right\} }^{\min \left\{ j,l\right\} }%
{\binom li}{\binom m{j-i}}a^{l-i}b^{m+i-j},
\end{equation}
and 
\begin{equation}
\overline{R_{p,x}^{\mu \nu }R_{p,x}^\mu }=R_{p,x}^{\mu \nu }-R_{p,x}^\mu .
\end{equation}
The final product takes the following form: 
\begin{equation}
\phi (l_1^\mu ,l_2^\mu ;R_{1,x}^\mu ,R_{2,x}^\mu )\phi (l_1^\nu ,l_2^\nu
;R_{1,x}^\nu ,R_{2,x}^\nu )=K_{\mu \nu }^x\sum_{i_1}^{l_1^\mu +l_1^\nu
}\sum_{i_2}^{l_2^\mu +l_2^\nu }f_{i_1}f_{i_2}\phi (i_1,i_2;R_{1,x}^{\mu \nu
},R_{2,x}^{\mu \nu }),
\end{equation}
where for simplicity, only the $x$-component has been shown.

\section{Analytical derivatives of molecular integrals}

To demonstrate the analytical differentiation of molecular integrals over
GG's, consider integrals of the following form: 
\begin{equation}
O_{\mu \nu }=\int \!\!\!\int \phi (l_1^\mu ,m_1^\mu ,n_1^\mu ,l_2^\mu
,m_2^\mu ,n_2^\mu ;{\bf R}_1^\mu ,{\bf R}_2^\mu ){\hat{O}}\phi (l_1^\nu
,m_1^\nu ,n_1^\nu ,l_2^\nu ,m_2^\nu ,n_2^\nu ;{\bf R}_1^\nu ,{\bf R}_2^\nu )d%
{\bf r}_1d{\bf r}_2.
\end{equation}
Suppose, that we would like to calculate derivatives of the above integral
with respect to $\xi ^\mu $, $(\xi =\alpha _1^\mu ,\alpha _2^\mu ,\beta
_{12}^\mu ,R_{1,x}^\mu \ldots )$. These can be obtained as follows: 
\begin{equation}
{\frac{\partial O_{\mu \nu }}{\partial \xi ^\mu }}=\int \!\!\!\int \left( {%
\frac \partial {\partial \xi ^\mu }}\phi (l_1^\mu ,m_1^\mu ,n_1^\mu ,l_2^\mu
,m_2^\mu ,n_2^\mu ;{\bf R}_1^\mu ,{\bf R}_2^\mu )\right) {\hat{O}}\phi
(l_1^\nu ,m_1^\nu ,n_1^\nu ,l_2^\nu ,m_2^\nu ,n_2^\nu ;{\bf R}_1^\nu ,{\bf R}%
_2^\nu )d{\bf r}_1d{\bf r}_2.
\end{equation}
This procedure can be applied to the overlap integral, nuclear attraction
integral and electron repulsion integral; however, it cannot be applied
directly to the kinetic energy. The kinetic energy integral must first be
expressed in terms of overlap integrals thereby permitting the
straightforward implementation of the above procedure.

\subsection{Overlap Integral}

The overlap integral over GG's can be expressed as a product of three
two-dimensional overlap integrals:

\begin{equation}
<\phi _\mu |\phi _\nu >\ =\ <l_1^\mu ,l_2^\mu |l_1^\nu ,l_2^\nu ><m_1^\mu
,m_2^\mu |m_1^\nu ,m_2^\nu ><n_1^\mu ,n_2^\mu |n_1^\nu ,n_2^\nu >,
\end{equation}
where these overlap integrals are represented as, 
\begin{equation}
<l_1^\mu ,l_2^\mu |l_1^\nu ,l_2^\nu >\ =\
<0,0|0,0>\sum_{i_1,i_2,r_1,r_2,u}A_{i_1,i_2,r_1,r_2,u},
\end{equation}
with the spherical component of the overlap integral expressed as, 
\begin{equation}
<0,0|0,0>\ =\ {\frac{{\ \pi K_{\mu \nu }^x}}{\sqrt{a_1^{\mu \nu }a_2^{\mu
\nu }+\beta ^{\mu \nu }(a_1^{\mu \nu }+a_2^{\mu \nu })}},}  \label{detab}
\end{equation}
where ${K_{\mu \nu }^x}$ is given in eq.(\ref{k}), and the Cartesian
components as, 
\[
A_{i_1,i_2,r_1,r_2,u}=(-)^{i_1+u}f_{i_1}(l_1^\mu ,l_1^\nu ,\overline{%
R_{1,x}^{\mu \nu }R_{1,x}^\mu },\overline{R_{1,x}^{\mu \nu }R_{1,x}^\nu }%
)f_{i_2}(l_1^\mu ,l_1^\nu ,\overline{R_{2,x}^{\mu \nu }R_{2,x}^\mu },%
\overline{R_{2,x}^{\mu \nu }R_{2,x}^\nu })C_{r_1}^{i_1}C_{r_2}^{i_2}\times 
\]
\begin{equation}
\left( {\frac{a_1^{\mu \nu }a_2^{\mu \nu }+\beta _{12}^{\mu \nu }(a_1^{\mu
\nu }+a_2^{\mu \nu })}{4a_1^{\mu \nu }a_2^{\mu \nu }\beta _{12}^{\mu \nu }}}%
\right) ^{2(r_1+r_2)+u-i_1-i_2}(\overline{R_{1,x}^{\mu \nu }R_{2,x}^{\mu \nu
}})^{i_1+i_2-2(r_1+r_2+u)}
\end{equation}
where, 
\begin{equation}
C_m^n={\frac{n!\ }{2^n\alpha ^{n-m}m!\ (n-2m)!\ }}.
\end{equation}
Using the prescriptions for derivatives of GG's presented previously, the
derivatives with respect to orbital exponents are obtained: 
\[
{\frac{{\partial <\phi _\mu |\phi _\nu >}}{{\ \partial \alpha _1^\mu }}}=
\]
\[
-[<l_1^\mu +2,l_2^\mu |l_1^\nu ,l_2^\nu ><m_1^\mu ,m_2^\mu |m_1^\nu ,m_2^\nu
><n_1^\mu ,n_2^\mu |n_1^\nu ,n_2^\nu >\ +
\]
\[
<l_1^\mu ,l_2^\mu |l_1^\nu ,l_2^\nu ><m_1^\mu +2,m_2^\mu |m_1^\nu ,m_2^\nu
><n_1^\mu ,n_2^\mu |n_1^\nu ,n_2^\nu >+
\]
\begin{equation}
<l_1^\mu ,l_2^\mu |l_1^\nu ,l_2^\nu ><m_1^\mu ,m_2^\mu |m_1^\nu ,m_2^\nu
><n_1^\mu +2,n_2^\mu |n_1^\nu ,n_2^\nu >],
\end{equation}
and in a similar fashion, derivatives with respect to the correlation
exponent, 
\[
{\frac{{\partial <\phi _\mu |\phi _\nu >}}{{\ \partial \beta _{12}^\mu }}}=
\]
\[
-\bigg[
<l_1^\mu +2,l_2^\mu |l_1^\nu ,l_2^\nu >\ -\ 2<l_1^\mu +1,l_2^\mu +1|l_1^\nu
,l_2^\nu >\ +\ <l_1^\mu ,l_2^\mu +2|l_1^\nu ,l_2^\nu >
\]
\[
\ +\ 2(R_{1,x}^\mu \ -\ R_{2,x}^\mu )<l_1^\mu +1,l_2^\mu |l_1^\nu ,l_2^\nu
>\ +\ 2(R_{2,x}^\mu \ -\ R_{1,x}^\mu )<l_1^\mu ,l_2^\mu +1|l_1^\nu ,l_2^\nu >
\]
\[
\ +\ (R_{1,x}^\mu -R_{2,x}^\mu )^2<l_1^\mu ,l_2^\mu |l_1^\nu ,l_2^\nu >%
\bigg]  
<m_1^\mu ,m_2^\mu |m_1^\nu ,m_2^\nu ><n_1^\mu ,n_2^\mu |n_1^\nu ,n_2^\nu >
\]
\begin{equation}
\ +\ y_{comp}\ +\ z_{comp}.
\end{equation}
Partial differentiation with respect to the orbital center can be performed
in the following manner: 
\[
{\frac{{\partial <\phi _\mu |\phi _\nu >}}{{\ \partial R_{1,x}^\mu }}}=
\]
\[
\lbrack 2\alpha _1^\mu <l_1^\mu +1,l_2^\mu |l_1^\nu ,l_2^\nu >-l_1^\mu
<l_1^\mu ,l_2^\mu |l_1^\nu ,l_2^\nu >]\times 
\]
\begin{equation}
<m_1^\mu ,m_2^\mu |m_1^\nu ,m_2^\nu ><n_1^\mu ,n_2^\mu |n_1^\nu ,n_2^\nu >.
\end{equation}
This method is extensible to derivatives with respect to other orbital
centers as well.

\subsection{Kinetic-Energy Integral}

The kinetic energy matrix element can be considered as a sum of two
components: 
\begin{equation}
<T^{\mu \nu }>\ =\ <T_1^{\mu \nu }>\ +\ <T_2^{\mu \nu }>
\end{equation}
where, the first component has the following form for example, 
\[
2<T_1^{\mu \nu }>\ =
\]
\begin{equation}
-\ \int \!\!\!\int \phi (l_1^\mu ,m_1^\mu ,n_1^\mu ,l_2^\mu ,m_2^\mu
,n_2^\mu ;{\bf R}_1^\mu ,{\bf R}_2^\mu ){\nabla _1^2}\phi (l_1^\nu ,m_1^\nu
,n_1^\nu ,l_2^\nu ,m_2^\nu ,n_2^\nu ;{\bf R}_1^\nu ,{\bf R}_2^\nu )d{\bf r}%
_1d{\bf r}_2.
\end{equation}
This last equation cannot be differentiated directly, as previously
discussed. We must first find an expression in terms of overlap integrals,
whose derivatives are easily managed. To demonstrate this, let us transform
the kinetic energy integral using Green's theorem \cite{les64}: 
\[
2<T_1^{\mu \nu }>\ =\int d{\bf r}_2G(l_2^\mu ,m_2^\mu ,n_2^\mu ,{\bf r}_2,%
{\bf R}_2^\mu ,\alpha _2^\mu )G(l_2^\nu ,m_2^\nu ,n_2^\nu ,{\bf r}_2,{\bf R}%
_2^\nu ,\alpha _2^\nu )\times 
\]
\[
\int d{\bf r}_1\{4\beta _{12}^\mu \beta _{12}^\nu r_{12}^2G(l_1^\mu ,m_1^\mu
,n_1^\mu ,{\bf r}_1,{\bf R}_1^\mu ,\alpha _1^\mu )G(l_1^\nu ,m_1^\nu
,n_1^\nu ,{\bf r}_1,{\bf R}_1^\nu ,\alpha _1^\nu )
\]
\begin{equation}
-(\beta _{12}^\mu +\beta _{12}^\nu )^{-1}W({\bf r}_1)\}\exp [-(\beta
_{12}^\mu +\beta _{12}^\nu )r_{12}^2],
\end{equation}
where 
\[
W({\bf r}_1)\ =\ 
\]
\[
\left( \frac{\beta _{12}^\nu }{\beta _{12}^\mu +\beta _{12}^\nu }\right)
G(l_1^\nu ,m_1^\nu ,n_1^\nu ,{\bf r}_1,{\bf R}_1^\nu ,\alpha _1^\nu ){\nabla
_1^2}G(l_1^\mu ,m_1^\mu ,n_1^\mu ,{\bf r}_1,{\bf R}_1^\mu ,\alpha _1^\mu )
\]
\begin{equation}
+\ \left( \frac{\beta _{12}^\mu }{\beta _{12}^\mu +\beta _{12}^\nu }\right)
G(l_1^\mu ,m_1^\mu ,n_1^\mu ,{\bf r}_1,{\bf R}_1^\mu ,\alpha _1^\mu ){\nabla
_1^2}G(l_1^\nu ,m_1^\nu ,n_1^\nu ,{\bf r}_1,{\bf R}_1^\nu ,\alpha _1^\nu ).
\end{equation}
This formula is valid for \{$\beta _{12}^\mu =0$, $\beta _{12}^\nu \neq 0$%
\}, or \{$\beta _{12}^\mu \neq 0$, $\beta _{12}^\nu =0$\}, and additionally
holds for \{$\beta _{12}^\mu =$ $\beta _{12}^\nu =0$\}. In the later case,
it can be shown that $\beta _{12}^\mu /(\beta _{12}^\mu +\beta _{12}^\nu )$
and $\beta _{12}^\nu /(\beta _{12}^\mu +\beta _{12}^\nu )$ must be replaced
by ($\frac 12$). The first term in the expression for the kinetic energy, as
shown above, can be expressed as: 
\[
2<T_1^{\mu \nu }>\ =\ 
\]
\[
4\beta _{12}^\mu \beta _{12}^\nu \bigg[
<l_1^\mu +2,l_2^\mu |l_1^\nu ,l_2^\nu >\ -\ 2<l_1^\mu +1,l_2^\mu +1|l_1^\nu
,l_2^\nu >\ +\ <l_1^\mu ,l_2^\mu +2|l_1^\nu ,l_2^\nu >
\]
\[
\ +\ 2(R_{1,x}^\mu \ -\ R_{2,x}^\mu )<l_1^\mu +1,l_2^\mu |l_1^\nu ,l_2^\nu
>\ +\ 2(R_{2,x}^\mu \ -\ R_{1,x}^\mu )<l_1^\mu ,l_2^\mu +1|l_1^\nu ,l_2^\nu >
\]
\[
\ +\ (R_{1,x}^\mu -R_{2,x}^\mu )^2<l_1^\mu ,l_2^\mu |l_1^\nu ,l_2^\nu >%
\bigg]  
<m_1^\mu ,m_2^\mu |m_1^\nu ,m_2^\nu ><n_1^\mu ,n_2^\mu |n_1^\nu ,n_2^\nu >
\]
\[
\ +\ 4\beta _{12}^\mu \beta _{12}^\nu y_{comp}\ +\ 4\beta _{12}^\mu \beta
_{12}^\nu z_{comp}
\]
\[
\int d{\bf r}_2G(l_2^\mu ,m_2^\mu ,n_2^\mu ,{\bf r}_2,{\bf R}_2^\mu ,\alpha
_2^\mu )G(l_2^\nu ,m_2^\nu ,n_2^\nu ,{\bf r}_2,{\bf R}_2^\nu ,\alpha _2^\nu
)\times 
\]
\begin{equation}
\int d{\bf r}_1\{-\ (\beta _{12}^\mu +\beta _{12}^\nu )^{-1}W({\bf r}%
_1)\}\exp [-(\beta _{12}^\mu +\beta _{12}^\nu )r_{12}^2],
\end{equation}
where for brevity, again only the $x$-component has been shown. One now
obtains: 
\[
{\nabla _1^2}G(l_1^\mu ,m_1^\mu ,n_1^\mu ,{\bf r}_1,{\bf R}_1^\mu ,\alpha
_1^\mu )\ =\ -2\alpha _1^\mu [2(l_1^\mu +m_1^\mu +n_1^\mu )+3]G(l_1^\mu
,m_1^\mu ,n_1^\mu ,{\bf r}_1,{\bf R}_1^\mu ,\alpha _1^\mu )
\]
\[
-4\alpha _1^\mu [G(l_1^\mu +2,m_1^\mu ,n_1^\mu ,{\bf r}_1,{\bf R}_1^\mu
,\alpha _1^\mu )\ +\ G(l_1^\mu ,m_1^\mu +2,n_1^\mu ,{\bf r}_1,{\bf R}_1^\mu
,\alpha _1^\mu )
\]
\[
\ +\ G(l_1^\mu ,m_1^\mu ,n_1^\mu +2,{\bf r}_1,{\bf R}_1^\mu ,\alpha _1^\mu
)]\ +\ \ l_1^\mu (l_1^\mu -1)G(l_1^\mu -2,m_1^\mu ,n_1^\mu ,{\bf r}_1,{\bf R}%
_1^\mu ,\alpha _1^\mu )
\]
\begin{equation}
+\ m_1^\mu (m_1^\mu -1)G(l_1^\mu ,m_1^\mu -2,n_1^\mu ,{\bf r}_1,{\bf R}%
_1^\mu ,\alpha _1^\mu )+\ n_1^\mu (n_1^\mu -1)G(l_1^\mu ,m_1^\mu ,n_1^\mu -2,%
{\bf r}_1,{\bf R}_1^\mu ,\alpha _1^\mu ).
\end{equation}
Differentiation of the kinetic integral can now be easily expressed in terms
of integrals of the overlap.

\subsection{Nuclear Attraction Integral}

The analytical derivatives of the nuclear attraction integrals and electron
repulsion integrals can be evaluated in similar fashion; however, for
economy of presentation, we will keep them in separate subsections.
According to Lester and Krauss \cite{les64}, the nuclear attraction integral
has the generalized form: 
\begin{equation}
<\phi _\mu |{\frac 1{|{\bf r}-{\bf R_E}|}}|\phi _\nu >\ =\ B_0\sum_{\{I\}}\
B_{\{I\}}\sum_{\{J\}}\ B_{\{J\}}\sum_{\{K\}}\ B_{\{K\}}F_\nu (Z_1)
\end{equation}
which can be denoted as $<l_1^\mu ,l_2^\mu ,m_1^\mu ,m_2^\mu ,n_1^\mu
,n_2^\mu |{\frac 1{|{\bf r}-{\bf R_E}|}}|l_1^\mu ,l_2^\mu ,m_1^\mu ,m_2^\mu
,n_1^\mu ,n_2^\mu >$. The derivative with respect to orbital exponents has
the following form: 
\[
{\frac \partial {{\ \partial \alpha _1^\mu }}}<\phi _\mu |{\frac 1{|{\bf r}-%
{\bf R_E}|}}|\phi _\nu >\ =\ 
\]
\[
<l_1^\mu +2,l_2^\mu ,m_1^\mu ,m_2^\mu ,n_1^\mu ,n_2^\mu |{\frac 1{|{\bf r}-%
{\bf R_E}|}}|l_1^\mu ,l_2^\mu ,m_1^\mu ,m_2^\mu ,n_1^\mu ,n_2^\mu >
\]
\[
<l_1^\mu ,l_2^\mu ,m_1^\mu +2,m_2^\mu ,n_1^\mu ,n_2^\mu |{\frac 1{|{\bf r}-%
{\bf R_E}|}}|l_1^\mu ,l_2^\mu ,m_1^\mu ,m_2^\mu ,n_1^\mu ,n_2^\mu >
\]
\begin{equation}
<l_1^\mu ,l_2^\mu ,m_1^\mu ,m_2^\mu ,n_1^\mu +2,n_2^\mu |{\frac 1{|{\bf r}-%
{\bf R_E}|}}|l_1^\mu ,l_2^\mu ,m_1^\mu ,m_2^\mu ,n_1^\mu ,n_2^\mu >
\end{equation}
Differentiation with respect to correlation exponents: 
\[
{\frac \partial {{\ \partial \beta _{12}^\mu }}}<\phi _\mu |{\frac 1{|{\bf r}%
-{\bf R_E}|}}|\phi _\nu >\ =\ 
\]
\[
+\ <l_1^\mu +2,l_2^\mu ,m_1^\mu ,m_2^\mu ,n_1^\mu ,n_2^\mu |{\frac 1{|{\bf r}%
-{\bf R_E}|}}|l_1^\nu ,l_2^\nu ,m_1^\nu ,m_2^\nu ,n_1^\nu ,n_2^\nu >
\]
\[
-\ 2<l_1^\mu +1,l_2^\mu +1,m_1^\mu ,m_2^\mu ,n_1^\mu ,n_2^\mu |{\frac 1{|%
{\bf r}-{\bf R_E}|}}|l_1^\nu ,l_2^\nu ,m_1^\nu ,m_2^\nu ,n_1^\nu ,n_2^\nu >
\]
\[
+\ <l_1^\mu ,l_2^\mu +2,m_1^\mu ,m_2^\mu ,n_1^\mu ,n_2^\mu |{\frac 1{|{\bf r}%
-{\bf R_E}|}}|l_1^\nu ,l_2^\nu ,m_1^\nu ,m_2^\nu ,n_1^\nu ,n_2^\nu >
\]
\[
\ +\ 2(R_{1,x}^\mu -R_{2,x}^\mu )<l_1^\mu +1,l_2^\mu ,m_1^\mu ,m_2^\mu
,n_1^\mu ,n_2^\mu |{\frac 1{|{\bf r}-{\bf R_E}|}}|l_1^\nu ,l_2^\nu ,m_1^\nu
,m_2^\nu ,n_1^\nu ,n_2^\nu >
\]
\[
\ +\ 2(R_{2,x}^\mu -R_{1,x}^\mu )<l_1^\mu ,l_2^\mu +1,m_1^\mu ,m_2^\mu
,n_1^\mu ,n_2^\mu |{\frac 1{|{\bf r}-{\bf R_E}|}}|l_1^\nu ,l_2^\nu ,m_1^\nu
,m_2^\nu ,n_1^\nu ,n_2^\nu >
\]
\begin{equation}
\ +\ (R_{1,x}^\mu -R_{2,x}^\mu )^2<l_1^\mu ,l_2^\mu ,m_1^\mu ,m_2^\mu
,n_1^\mu ,n_2^\mu |{\frac 1{|{\bf r}-{\bf R_E}|}}|l_1^\nu ,l_2^\nu ,m_1^\nu
,m_2^\nu ,n_1^\nu ,n_2^\nu >+y_{comp}+z_{comp}
\end{equation}
The derivative with respect to orbital centers can now be obtained as: 
\[
{\frac \partial {{\ \partial R_{1,x}^\mu }}}<\phi _\mu |{\frac 1{|{\bf r}-%
{\bf R_E}|}}|\phi _\nu >\ =\ 
\]
\[
2\alpha _1^\mu <l_1^\mu +1,l_2^\mu ,m_1^\mu ,m_2^\mu ,n_1^\mu ,n_2^\mu |{%
\frac 1{|{\bf r}-{\bf R_E}|}}|l_1^\mu ,l_2^\mu ,m_1^\mu ,m_2^\mu ,n_1^\mu
,n_2^\mu >
\]
\begin{equation}
-l_1^\mu <l_1^\mu -1,l_2^\mu ,m_1^\mu ,m_2^\mu ,n_1^\mu ,n_2^\mu |{\frac 1{|%
{\bf r}-{\bf R_E}|}}|l_1^\mu ,l_2^\mu ,m_1^\mu ,m_2^\mu ,n_1^\mu ,n_2^\mu >.
\end{equation}

\subsection{Electron Repulsion Integral}

\begin{equation}
<\phi _\mu |{\frac 1{|{\bf r}_2-{\bf r}_1|}}|\phi _\nu >\ =\ \sum_{\{I\}}\
C_{\{I\}}\sum_{\{J\}}\ C_{\{J\}}\sum_{\{K\}}\ C_{\{K\}}F_\nu (Z),
\end{equation}
can be recast into a representation similar to the nuclear attraction
integral: 
\begin{equation}
<l_1^\mu ,l_2^\mu ,m_1^\mu ,m_2^\mu ,n_1^\mu ,n_2^\mu |{\frac 1{|{\bf r}_2-%
{\bf r}_1|}}|l_1^\mu ,l_2^\mu ,m_1^\mu ,m_2^\mu ,n_1^\mu ,n_2^\mu >.
\end{equation}
The formulae derived above are quite valid for ER integrals after the
appropriate exchange of ${\frac 1{|{\bf r}-{\bf R_E}|}}$ for ${\frac 1{|{\bf %
r}_2-{\bf r}_1|}}$ . Such a procedure should be straightforward and need not
be redundantly stated here.

\section{Numerical Results}

A simple implementation of a gradient-based calculation for molecular
hydrogen was performed. The preliminary results served to motivate the
derivation of the above generalized formulae for analytic derivatives in the
GG\ basis. In this section, we will briefly describe how this
gradient-optimization of the ground-state of H$_2$ was performed in a
multi-center GG basis.

The form of the spatial part of the ground state wavefunction used in the
present calculations (properly symmetrized with respect to exchange of
identical particles) was: 
\begin{equation}
\Psi (1,2)=\sum\limits_{\mu =1}^nc_\mu (1+P_{12})(1+P_{H_1H_2})\phi _\mu
(0,0,0,0,0,0;R_A^\mu ,R_B^\mu ),
\end{equation}
where 
\begin{equation}
\phi _\mu (0,0,0,0,0,0;R_A^\mu ,R_B^\mu )=e^{\alpha _1^\mu \left|
r_1-R_A^\mu \right| ^2-\alpha _2^\mu \left| r_2-R_B^\mu \right| ^2-\beta
_{12}^\mu \left| r_1-r_2\right| ^2}
\end{equation}
[see eq.(\ref{eq3})],$\;P_{12}$ represents exchange of electrons, and $%
P_{H_1H_2}$ represents exchange of nuclei. $R_A$ and $R_B$ are not
necessarily assumed to represent the centers of the nuclei ($R_{H1},R_{H2}$%
). Nor should it be assumed that $R_A=R_B$.

The analytical derivatives of the GG molecular integrals were realized via
''automatic differentiation'' of Fortran90 program code with the aid of
readily available software packages. (A paper anticipated to detail the
procedure of ''automatic differentiation'', among other items of interest,
should reach publication some time hence\cite{gilmore1}). The components of
the analytical derivative were rigorously tested by finite differencing of
the original program code.

Optimization occurs via minimization of the Rayleigh quotient:
\begin{equation}
E_{Ground\ State}=\stackunder{\left\{ c,a\right\} }{\min }(E_{Rayleigh}),
\end{equation}
\begin{equation}
E_{Rayleigh}=\frac{\left\langle \Psi (1,2)\right| \widehat{H}\left| \Psi
(1,2)\right\rangle }{\left\langle \Psi (1,2)\right| \left. \Psi
(1,2)\right\rangle }=\frac{c^{\prime }H\left( a\right) c}{c^{\prime }S\left(
a\right) c},
\end{equation}
with respect to linear and non-linear parameters, where the prime ($^{\prime
}$) indicates transposition, $c$\ represents the linear parameters, $a$ the
set of all non-linear optimization parameters: $\left\{ \alpha _1^\mu
,\alpha _2^\mu ,\beta _{12}^\mu ,R_1^\mu ,R_2^\mu \right\} $, $H$\ the
Hamiltonian matrix, and $S$ the overlap matrix. The optimization is
constrained by requiring that each GG is a square-integrable function; which
condition is fulfilled when the determinant: 
\begin{equation}
det\left( 
\begin{array}{ll}
\alpha _1+\beta _{12} & -\beta _{12} \\ 
-\beta _{12} & \alpha _2+\beta _{12}
\end{array}
\right) =a_1^\mu a_2^\mu +\beta _{12}^\mu (a_1^\mu +a_2^\mu ),
\end{equation}
remains positive (see eq.\ref{detab}). This is accomplished by utilizing the
following relation: $\det (tt^{\prime })>0$, which is true for any
lower-triangular, full-rank matrix 
\[
t=\left( 
\begin{array}{ll}
t_1 & 0 \\ 
t_3 & t_2
\end{array}
\right) ,
\]
whose elements are not constrained (except for $t_1\neq 0$ and $t_2\neq 0$).
This requires the following Choelesky decomposition: 
\[
tt^{\prime }=\left( 
\begin{array}{ll}
\alpha _1+\beta _{12} & -\beta _{12} \\ 
-\beta _{12} & \alpha _2+\beta _{12}
\end{array}
\right) .
\]
From this equation, we obtain the following relations between $\alpha
_1,\alpha _2,\beta _{12}$ and $t_1,t_2,t_3$: 
\begin{equation}
\alpha _1=t_1^2+t_1t_3
\end{equation}
\begin{equation}
\alpha _2=t_2^2+t_3^2+t_1t_3
\end{equation}
\begin{equation}
\beta _{12}=-t_1t_3.
\end{equation}
Optimization proceeds with the substitution of the unconstrained variables \{%
$t_1^\mu ,t_2^\mu ,t_3^\mu $\} for the constrained ones \{$\alpha _1^\mu
,\alpha _2^\mu ,\beta _{12}^\mu $\}.

The differential with respect to the linear coefficients is given as: 
\begin{equation}
\partial E_{linear}=\frac{2c^{\prime }(H-ES)\partial c}{c^{\prime }Sc}.
\end{equation}
The differential with respect to the non-linear parameters is given as: 
\begin{equation}
\partial E_{non-linear}=\frac{2c^{\prime }(\partial H-E\partial S)c}{%
c^{\prime }Sc}=(c\otimes c)^{\prime }\left( \frac{\partial vecH}{\partial
a^{\prime }}-E\frac{\partial vecS}{\partial a^{\prime }}\right) \partial a,
\label{nonlinearE}
\end{equation}
where the Kronecker product ($\otimes $) and the matrix vectorization
operator ($vec$) were utilized as described by Kinghorn\cite{kinghorn2}. The
non-zero terms in the matrix $\partial vecH/\partial a^{\prime }$ are the
partial derivatives with respect to all of the non-linear parameters in the
basis (e.g., components such as $\partial T_{\mu \nu }/\partial \alpha
_1^\mu $). The right-hand-side of eq.(\ref{nonlinearE}) contains sparse
matrices which were re-assembled into a dense-matrix data structure\cite
{kinghorn1} and numerically verified.

Several optimization algorithms were investigated, as is required by most
non-linear optimization problems. The method of choice was a
''stochastic-growth, gradient minimization'' algorithm\cite{gilmore1} in
which the GG\ basis set was systematically grown from $n=1$ to $n=140$. The
IMSL\ DUMING, 'double precision, multi-variate, quasi-newton, analytic
gradient based' optimization routine was employed with energy functional
minimization proceeding to double precision, machine-specific convergence
criteria.

When the basis set is small, calculations proceed quite rapidly. When the
basis set starts to become large, it has previously gone through several
optimization iterations and is considered to be quite well optimized
already. The result of this is that subsequent optimization steps require
fewer iterations to achieve the convergence criteria.

\medskip \medskip 

The best GG literature result at present is -1.174475931 obtained with 700
functions\cite{rych1} and is represented by the horizontal line to which our
convergence profile appears asymptotic in {\it Figure 1}. The results
reported in {\it Table 1} represent 17 of the 141 stochastic-growth
iterations which were performed. These points were selected based upon the
consideration of having converged very well relative to the remaining
iterations.

We do not report a new bound on the GG\ calculation of molecular hydrogen.
Instead, we have presented a rudimentary implementation of an
analytic-gradient optimization which preliminarily shows a rapidly
convergent method of conducting such calculations. To that end, we have
provided the necessary generalized formulae for analytic first derivatives
in the multi-center GG\ basis. Proposed future efforts would include
refinement of this procedure to improve efficiency for large basis set size.
An analytic form for the Hessian would be of important utility should basis
set term-coupling exist, and would be expected to greatly enhance overall
convergence as well. Successful achievement of these goals would
significantly contribute to the application of this general methodology to
n-particle systems and systems with higher angular momentum.

\subsection{Acknowledgments}

This study was supported by the National Science Foundation under the grant
No. CHE-9300497

\begin{thebibliography}{99}
\bibitem{hyl29}  E. A. Hylleraas {\it Z. Phys.} {\bf 54}, 347 (1929).

\bibitem{jam33}  H. M. James and A. S. Coolidge, {\it J. Chem. Phys.} {\bf 1}%
, 825 (1933).

\bibitem{kol65}  W. Kolos and L. Wolniewicz, {\it J. Chem. Phys.} {\bf 43},
2429 (1965).

\bibitem{fromm1}  D. M. Fromm and R. N. Hill, {\it Phys. Rev. }A {\bf 36},
1013 (1987).

\bibitem{king1}  F. W. King, {\it Phys. Rev. }A {\bf 44}, 7108 (1991).

\bibitem{boy60}  S. F. Boys, {\it Proc. R. Soc.} London Ser. {\bf A258}, 402
(1960).

\bibitem{sin60}  K. Singer, {\it Proc. R. Soc.} London Ser. {\bf A258}, 412
(1960).

\bibitem{les64}  W. A. Lester, Jr. and M. Krauss, {\it J. Chem. Phys.} {\bf %
41}, 1407 (1964); {\bf 41}, 2990 (1965).

\bibitem{koz91}  P. M. Kozlowski and L. Adamowicz, {\it J. Chem. Phys.} {\bf %
95}, 6681 (1991).

\bibitem{koz92a}  P. M. Kozlowski and L. Adamowicz, {\it J. Compt. Chem.} 
{\bf 13}, 602 (1992).

\bibitem{sza83}  K. Szalewicz, B. Jeziorski, H. J. Monkhorst and J. G.
Zabolitzky, {\it J. Chem. Phys.} {\bf 78}, 1420 (1983).

\bibitem{koz93a}  P. M. Kozlowski and L. Adamowicz, {\it Phys. Rev.} A {\bf %
48}, 1903 (1993).

\bibitem{koz93}  P. M. Kozlowski and L. Adamowicz, {\it Chem. Rev.} {\bf 93}%
, 2007 (1993).

\bibitem{kom93}  J. Komasa and A. J. Thakkar, {\it Mol. Phys.} {\bf 78},
1093 (1993).

\bibitem{cen93}  W. Cencek and J. Rychlewski, {\it J. Chem. Phys.} {\bf 98},
1253 (1993).

\bibitem{liu93}  J. W. Liu and S. Hagstrom, {\it Phys. Rev. A} {\bf 48}, 166
(1993).

\bibitem{kol94}  W. Kolos, {\it J. Chem. Phys.} {\bf 101}, 1330 (1994).

\bibitem{powell1}  M. J. D. Powell, {\it Mathematical Programming } {\bf 29}%
, 297 (1977).

\bibitem{poshusta1}  R. D. Poshusta, {\it J. Quantum Chem., Quantum Chem.
Symposium }{\bf 13}, 59 (1979).

\bibitem{Alexander1}  S. A. Alexander, H. J. Monkhorst, and K. Szalewicz, 
{\it J. Chem. Phys. }{\bf 85}, 5821 (1986).

\bibitem{Alexander2}  S. A. Alexander, H. J. Monkhorst, and R. Roeland, {\it %
J. Chem. Phys. }{\bf 93}, 4230 (1990).

\bibitem{Rybak}  S. Rybak and K. Szalewicz, {\it J. Chem. Phys. }{\bf 91},
4479 (1989).

\bibitem{koz92b}  P. M. Kozlowski and L. Adamowicz, {\it J. Chem. Phys.} 
{\bf 96}, 9013 (1992).

\bibitem{kinghorn1}  D. B. Kinghorn, {\it Int. J. Quantum Chem.}, in press

\bibitem{gilmore1}  D. Gilmore, D. B. Kinghorn, and L. Adamowicz, to be
published.

\bibitem{kinghorn2}  D. B. Kinghorn, {\it Int. J. Quantum Chem.}, {\bf 57},
141-155 (1996).

\bibitem{rych1}  J. Rychlewski et al., {\it Chem. Phys. Let.,} {\bf 229},
659 (1994).
\end{thebibliography}

\enddocument

\end{document}
